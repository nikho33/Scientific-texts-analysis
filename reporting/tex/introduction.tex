%\chapter{Introduction}

Disclaimer: This report and especially this introduction is widely inspired by the review of Allahyari et al., \textit{A Brief Survey of Text Mining: Classification, Clustering and Extraction Techniques} \cite{Allahyari2017}.
\newline

In a very recent paper \cite{Tshitoyan2019}, researchers at Lawrence Berkeley National Laboratory have developed an artificial intelligence (AI) that, with training, has predicted discoveries in material science.
%\To spot what scientists had missed, all the AI had to do was read millions of previously published scientific papers. 
The scientists gathered 3.3 million abstracts on materials science from 1,000 different journals published between 1922 and 2018. 
They used the renowned algorithm Word2vec and the magic happened when fed abstracts published up to the year 2008, Word2vec was able to predict materials that appear in later studies.
%\It took 500,000 distinct words from those abstracts and built mathematical connections between them. And that gave it very intriguing powers of prediction.
%\Based on the literature it analyzed, the AI was able to determine which material has the best thermoelectric properties. But it did something even more extraordinary. When fed abstracts published up to the year 2008, Word2vec was able to predict materials that appear in later studies.
\newline

%\The amount of text that is generated every day is increasing dramatically. This tremendous volume of mostly unstructured text cannot be simply processed and perceived by computers.
%\The main source of machine-interpretable data for the materials research community has come from structured property databases. Beyond property values, publications contain valuable knowledge regarding the connections and relationships between data items as interpreted by the authors. To improve the identification and use of this knowledge, several studies have focused on the retrieval of information from scientific literature using supervised natural language processing, which requires large hand-labelled datasets for training.



