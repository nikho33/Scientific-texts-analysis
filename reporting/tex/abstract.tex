
\begin{abstract}
	The project is part of the Artificial Intelligence (AI) and Machine Learning (ML) training course held at Sorbonne University (Paris, France) during 2018-2019 academic year. As final project, a work on the analysis and classification of scientific texts have been chosen. 
	\newline
	The initial motivations were to discover and learn mathematical algorithms and methods used in AI and ML in order to evaluate any application for the Fluigent European project HoliFAB. It aims at adapting an existing pilot line for the production of microfluidic instruments, and develops hardware and software strategies for the optimization of the production and system. 
	Some mathematical developments at the beginning of the project led to the implementation of regression functions that auto-place and auto-wire a system layout.
	\newline
	\newline
	However the motivation for the current project raises from the frustration as a researcher to not be able to read and study all papers relevant of a field. Making a bibliography on a specific topic is common and easy to perform thanks to different database and search engine available on the internet. However, when it comes to study a broad scientific field for a global understanding or new research investigation or market analysis, the task often lacks an intensive study of the domain. 
	\newline
	\newline
	Text mining, which is the task of extracting meaningful information from text, is developed here on a database of scientific papers. More precisely Latent Dirichlet Algorithms (LDA) are studied in order statistically categorize text and extract topics. The tool helps to find and name topics and applied on different database of different years check for evolution in the scientific research and next innovation that will probably lead the market.
  	\newline
  	\newline
  	\textbf{Keywords:} Text mining, classification, clustering, information extraction, topic extraction, information retrieval, Latent Dirichlet Algorithm (LDA)
\end{abstract}